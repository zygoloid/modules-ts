%!TEX root = ts.tex

\setcounter{chapter}{10}
\rSec0[over]{Overloading}

\setcounter{section}{4}
\rSec1[over.oper]{Overloaded operators}

\setcounter{subsection}{7}
\rSec2[over.literal]{User-defined literals}

\noindent
Modify paragraph 11.5.8/7 as follows:
\begin{std.txt}
    \pnum[7]
    \enternote
    Literal operators and literal operator templates are usually invoked 
    implicitly through user-defined literals (5.13.8). However, except for 
    the constraints described above, they are ordinary namespace-scope 
    functions and function templates. In particular, they are looked up 
    like ordinary functions and function templates and they follow the same 
    overload resolution rules. Also, they can be declared \tcode{inline}
     or \tcode{constexpr}, 
    they can have internal\added{, module,} or external linkage, they can be called explicitly, 
    their addresses can be taken, etc.
    \exitnote
\end{std.txt}

%% main file for the C++ standard.
%%

%%--------------------------------------------------
%% basics
\documentclass[ebook,10pt,oneside,openany,final]{memoir}

\usepackage[american]
           {babel}        % needed for iso dates
\usepackage[iso,american]
           {isodate}      % use iso format for dates
\usepackage[final]
           {listings}     % code listings
\usepackage{longtable}    % auto-breaking tables
\usepackage{ltcaption}    % fix captions for long tables
\usepackage{booktabs}     % fancy tables
\usepackage{relsize}      % provide relative font size changes
\usepackage{underscore}   % remove special status of '_' in ordinary text
\usepackage{verbatim}     % improved verbatim environment
\usepackage{parskip}      % handle non-indented paragraphs "properly"
\usepackage{array}        % new column definitions for tables
\usepackage[normalem]{ulem}
\usepackage{color}        % define colors for strikeouts and underlines
\usepackage{amsmath}      % additional math symbols
\usepackage{mathrsfs}     % mathscr font
\usepackage{microtype}
\usepackage{multicol}
\usepackage{xspace}
\usepackage{fixme}
\usepackage{lmodern}
\usepackage[T1]{fontenc}
\usepackage[pdftex, final]{graphicx}
\usepackage[pdftex,
            pdftitle={C++ Technical Specification},
            pdfsubject={C++ Technical Specification},
            pdfcreator={Gabriel Dos~Reis},
            bookmarks=true,
            bookmarksnumbered=true,
            pdfpagelabels=true,
            pdfpagemode=UseOutlines,
            pdfstartview=FitH,
            linktocpage=true,
            colorlinks=true,
            linkcolor=blue,
            plainpages=false
           ]{hyperref}
\usepackage{memhfixc}     % fix interactions between hyperref and memoir
\usepackage{xstring}
\usepackage{tcolorbox}
\usepackage{comment}
\usepackage[active,header=false,handles=false,copydocumentclass=false,generate=cxx20.ext]{extract}
\tcbuselibrary{breakable}

%!TEX root = ts.tex
%% layout.tex -- set overall page appearance

%%--------------------------------------------------
%%  set page size, type block size, type block position

\setstocksize{11in}{8.5in}
\settrimmedsize{11in}{8.5in}{*}
\setlrmarginsandblock{1in}{1in}{*}
\setulmarginsandblock{1in}{*}{1.618}

%%--------------------------------------------------
%%  set header and footer positions and sizes

\setheadfoot{\onelineskip}{2\onelineskip}
\setheaderspaces{*}{2\onelineskip}{*}

%%--------------------------------------------------
%%  make miscellaneous adjustments, then finish the layout
\setmarginnotes{7pt}{7pt}{0pt}
\checkandfixthelayout

%%--------------------------------------------------
%% Paragraph and bullet numbering

\newcounter{Paras}
\counterwithin{Paras}{chapter}
\counterwithin{Paras}{section}
\counterwithin{Paras}{subsection}
\counterwithin{Paras}{subsubsection}
\counterwithin{Paras}{paragraph}
\counterwithin{Paras}{subparagraph}

\newcounter{Bullets1}[Paras]
\newcounter{Bullets2}[Bullets1]
\newcounter{Bullets3}[Bullets2]
\newcounter{Bullets4}[Bullets3]

\makeatletter
\newcommand{\parabullnum}[2]{%
\stepcounter{#1}%
\noindent\makebox[0pt][l]{\makebox[#2][r]{%
\scriptsize\raisebox{.7ex}%
{%
\ifnum \value{Paras}>0
\ifnum \value{Bullets1}>0 (\fi%
                          \arabic{Paras}%
\ifnum \value{Bullets1}>0 .\arabic{Bullets1}%
\ifnum \value{Bullets2}>0 .\arabic{Bullets2}%
\ifnum \value{Bullets3}>0 .\arabic{Bullets3}%
\fi\fi\fi%
\ifnum \value{Bullets1}>0 )\fi%
\fi%
}%
\hspace{\@totalleftmargin}\quad%
}}}
\makeatother

\newcommand{\pnum}[1][]{\ifthenelse{\equal{#1}{}}{}{\setcounter{Paras}{#1}\addtocounter{Paras}{-1}}\parabullnum{Paras}{0pt}\ignorespaces}

% Leave more room for section numbers in TOC
\cftsetindents{section}{1.5em}{3.0em}

\lstnewenvironment{grammar}
  {\lstset{basicstyle=\rmfamily,columns=fullflexible,escapechar=@}}
  {}


%% This is used to typeset section headings of ISO C++ standard text, e.g.
%%   x.y.x Foo bar                   [foo.bar]
\newcounter{myclause}
\newcounter{mysection}[myclause]
\newcounter{mysubsection}[mysection]
\makeatletter
\renewcommand{\themysection}%
   {\arabic{myclause}}
\renewcommand{\themysection}%
   {\themyclause.\arabic{mysection}}
\renewcommand{\themysubsection}%
   {\themyclause.\themysection.\arabic{mysubsection}}
\newcommand{\mysection}[2]{\@startsection{mysection}{2}{\z@}%
  {-3.25ex \@plus -1ex \@minus -.2ex}%
  {1.ex \@plus.2ex}%
  {\normalfont\bfseries}[#1]{#1 \hfill [#2]}}
\makeatother

\newenvironment{std.txt}%
  {\begin{quote}\fontsize{9pt}{10pt}\fontfamily{pbk}\selectfont}%
  {\end{quote}}

\input{styles}
%!TEX root = std.tex
% Definitions and redefinitions of special commands

%%--------------------------------------------------
%% Difference markups
\definecolor{addclr}{rgb}{0,.6,.6}
\definecolor{remclr}{rgb}{1,0,0}
\definecolor{noteclr}{rgb}{0,0,1}

\renewcommand{\added}[1]{\textcolor{addclr}{\uline{#1}}}
\newcommand{\removed}[1]{\textcolor{remclr}{\sout{#1}}}
\renewcommand{\changed}[2]{\removed{#1}\added{#2}}

\newcommand{\nbc}[1]{[#1]\ }
\newcommand{\addednb}[2]{\added{\nbc{#1}#2}}
\newcommand{\removednb}[2]{\removed{\nbc{#1}#2}}
\newcommand{\changednb}[3]{\removednb{#1}{#2}\added{#3}}
\newcommand{\remitem}[1]{\item\removed{#1}}

\newcommand{\ednote}[1]{\textcolor{noteclr}{[Editor's note: #1] }}
% \newcommand{\ednote}[1]{}

\newenvironment{addedblock}
{
\color{addclr}
}
{
\color{black}
}
\newenvironment{removedblock}
{
\color{remclr}
}
{
\color{black}
}

%%--------------------------------------------------
%% Sectioning macros.  
% Each section has a depth, an automatically generated section 
% number, a name, and a short tag.  The depth is an integer in 
% the range [0,5].  (If it proves necessary, it wouldn't take much
% programming to raise the limit from 5 to something larger.)


% The basic sectioning command.  Example:
%    \Sec1[intro.scope]{Scope}
% defines a first-level section whose name is "Scope" and whose short
% tag is intro.scope.  The square brackets are mandatory.
\def\Sec#1[#2]#3{%
\ifcase#1\let\s=\chapter
      \or\let\s=\section
      \or\let\s=\subsection
      \or\let\s=\subsubsection
      \or\let\s=\paragraph
      \or\let\s=\subparagraph
      \fi%
\s[#3]{#3\hfill[#2]}\label{#2}}

% A convenience feature (mostly for the convenience of the Project
% Editor, to make it easy to move around large blocks of text):
% the \rSec macro is just like the \Sec macro, except that depths 
% relative to a global variable, SectionDepthBase.  So, for example,
% if SectionDepthBase is 1,
%   \rSec1[temp.arg.type]{Template type arguments}
% is equivalent to
%   \Sec2[temp.arg.type]{Template type arguments}
\newcounter{SectionDepthBase}
\newcounter{scratch}

\def\rSec#1[#2]#3{%
\setcounter{scratch}{#1}
\addtocounter{scratch}{\value{SectionDepthBase}}
\Sec{\arabic{scratch}}[#2]{#3}}

\newcommand{\synopsis}[1]{\textbf{#1}}

%%--------------------------------------------------
% Indexing

% locations
\newcommand{\indextext}[1]{\index[generalindex]{#1}}
\newcommand{\indexlibrary}[1]{\index[libraryindex]{#1}}
\newcommand{\indexgram}[1]{\index[grammarindex]{#1}}
\newcommand{\indeximpldef}[1]{\index[impldefindex]{#1}}

\newcommand{\indexdefn}[1]{\indextext{#1}}
\newcommand{\indexgrammar}[1]{\indextext{#1}\indexgram{#1}}
\newcommand{\impldef}[1]{\indeximpldef{#1}implementation-defined}

% appearance
\newcommand{\idxcode}[1]{#1@\tcode{#1}}
\newcommand{\idxhdr}[1]{#1@\tcode{<#1>}}
\newcommand{\idxgram}[1]{#1@\textit{#1}}

%%--------------------------------------------------
% General code style
\newcommand{\CodeStyle}{\ttfamily}
\newcommand{\CodeStylex}[1]{\texttt{#1}}

% Code and definitions embedded in text.
\newcommand{\tcode}[1]{\CodeStylex{#1}}
\newcommand{\techterm}[1]{\textit{#1}\xspace}
\newcommand{\defnx}[2]{\indexdefn{#2}\textit{#1}\xspace}
\newcommand{\defn}[1]{\defnx{#1}{#1}}
\newcommand{\term}[1]{\textit{#1}\xspace}
\newcommand{\grammarterm}[1]{\textit{#1}\xspace}
\newcommand{\placeholder}[1]{\textit{#1}}
\newcommand{\placeholdernc}[1]{\textit{#1\nocorr}}

%%--------------------------------------------------
%% allow line break if needed for justification
\newcommand{\brk}{\discretionary{}{}{}}
%  especially for scope qualifier
\newcommand{\colcol}{\brk::\brk}

%%--------------------------------------------------
%% Macros for funky text
\newcommand{\Cpp}{\texorpdfstring{C\kern-0.05em\protect\raisebox{.35ex}{\textsmaller[2]{+\kern-0.05em+}}}{C++}\xspace}
\newcommand{\CppIII}{\Cpp 2003\xspace}
\newcommand{\CppXI}{\Cpp 2011\xspace}
\newcommand{\CppXIV}{\Cpp 2014\xspace}
\newcommand{\opt}{{\ensuremath{_\mathit{opt}}}\xspace}
\newcommand{\shl}{<{<}}
\newcommand{\shr}{>{>}}
\newcommand{\dcr}{-{-}}
\newcommand{\exor}{\^{}}
\newcommand{\bigoh}[1]{\ensuremath{\mathscr{O}(#1)}}

% Make all tildes a little larger to avoid visual similarity with hyphens.
% FIXME: Remove \tilde in favour of \~.
\renewcommand{\tilde}{\textasciitilde}
\renewcommand{\~}{\textasciitilde}
\let\OldTextAsciiTilde\textasciitilde
\renewcommand{\textasciitilde}{\protect\raisebox{-0.17ex}{\larger\OldTextAsciiTilde}}

%%--------------------------------------------------
%% States and operators
\newcommand{\state}[2]{\tcode{#1}\ensuremath{_{#2}}}
\newcommand{\bitand}{\ensuremath{\, \mathsf{bitand} \,}}
\newcommand{\bitor}{\ensuremath{\, \mathsf{bitor} \,}}
\newcommand{\xor}{\ensuremath{\, \mathsf{xor} \,}}
\newcommand{\rightshift}{\ensuremath{\, \mathsf{rshift} \,}}
\newcommand{\leftshift}[1]{\ensuremath{\, \mathsf{lshift}_#1 \,}}

%% Notes and examples
\newcommand{\EnterBlock}[1]{[\,\textit{#1:}\space}
\newcommand{\ExitBlock}[1]{\textit{\,---\,end #1}\,]\xspace}
\newcommand{\enternote}{\EnterBlock{Note}}
\newcommand{\exitnote}{\ExitBlock{note}}
\newcommand{\enterexample}{\EnterBlock{Example}}
\newcommand{\exitexample}{\ExitBlock{example}}
\newenvironment{example}[1][Example]{\EnterBlock{#1}}{\ExitBlock{example}}
\newenvironment{note}[1][Note]{\enternote}{\exitnote}

%% Library function descriptions
\newcommand{\Fundescx}[1]{\textit{#1}\xspace}
\newcommand{\Fundesc}[1]{\Fundescx{#1:}}
\newcommand{\required}{\Fundesc{Required behavior}}
\newcommand{\requires}{\Fundesc{Requires}}
\newcommand{\effects}{\Fundesc{Effects}}
\newcommand{\postconditions}{\Fundesc{Postconditions}}
\newcommand{\postcondition}{\Fundesc{Postcondition}}
\newcommand{\preconditions}{\requires}
\newcommand{\precondition}{\requires}
\newcommand{\returns}{\Fundesc{Returns}}
\newcommand{\throws}{\Fundesc{Throws}}
\newcommand{\default}{\Fundesc{Default behavior}}
\newcommand{\complexity}{\Fundesc{Complexity}}
\newcommand{\remark}{\Fundesc{Remark}}
\newcommand{\remarks}{\Fundesc{Remarks}}
%\newcommand{\note}{\remark}
\newcommand{\notes}{\remarks}
\newcommand{\realnote}{\Fundesc{Note}}
\newcommand{\realnotes}{\Fundesc{Notes}}
\newcommand{\errors}{\Fundesc{Error conditions}}
\newcommand{\sync}{\Fundesc{Synchronization}}
\newcommand{\implimits}{\Fundesc{Implementation limits}}
\newcommand{\replaceable}{\Fundesc{Replaceable}}
\newcommand{\returntype}{\Fundesc{Return type}}
\newcommand{\cvalue}{\Fundesc{Value}}
\newcommand{\ctype}{\Fundesc{Type}}
\newcommand{\ctypes}{\Fundesc{Types}}
\newcommand{\dtype}{\Fundesc{Default type}}
\newcommand{\ctemplate}{\Fundesc{Class template}}
\newcommand{\templalias}{\Fundesc{Alias template}}

%% Cross reference
\newcommand{\xref}{\textsc{See also:}\xspace}
\newcommand{\xsee}{\textsc{See:}\xspace}
\newcommand{\stdref}[2]{#2}

%% NTBS, etc.
\newcommand{\NTS}[1]{\textsc{#1}\xspace}
\newcommand{\ntbs}{\NTS{ntbs}}
\newcommand{\ntmbs}{\NTS{ntmbs}}
\newcommand{\ntwcs}{\NTS{ntwcs}}
\newcommand{\ntcxvis}{\NTS{ntc16s}}
\newcommand{\ntcxxxiis}{\NTS{ntc32s}}

%% Code annotations
\newcommand{\EXPO}[1]{\textit{#1}}
\newcommand{\expos}{\EXPO{exposition only}}
\newcommand{\impdef}{\EXPO{implementation-defined}}
\newcommand{\impdefnc}{\EXPO{implementation-defined\nocorr}}
\newcommand{\impdefx}[1]{\indeximpldef{#1}\EXPO{implementation-defined}}
\newcommand{\notdef}{\EXPO{not defined}}

\newcommand{\UNSP}[1]{\textit{\texttt{#1}}}
\newcommand{\UNSPnc}[1]{\textit{\texttt{#1}\nocorr}}
\newcommand{\unspec}{\UNSP{unspecified}}
\newcommand{\unspecnc}{\UNSPnc{unspecified}}
\newcommand{\unspecbool}{\UNSP{unspecified-bool-type}}
\newcommand{\seebelow}{\UNSP{see below}}
\newcommand{\seebelownc}{\UNSPnc{see below}}
\newcommand{\unspecuniqtype}{\UNSP{unspecified unique type}}
\newcommand{\unspecalloctype}{\UNSP{unspecified allocator type}}

\newcommand{\EXPLICIT}{\textit{\texttt{EXPLICIT}}}

%% Manual insertion of italic corrections, for aligning in the presence
%% of the above annotations.
\newlength{\itcorrwidth}
\newlength{\itletterwidth}
\newcommand{\itcorr}[1][]{%
 \settowidth{\itcorrwidth}{\textit{x\/}}%
 \settowidth{\itletterwidth}{\textit{x\nocorr}}%
 \addtolength{\itcorrwidth}{-1\itletterwidth}%
 \makebox[#1\itcorrwidth]{}%
}

%% Double underscore
\newcommand{\ungap}{\kern.5pt}
\newcommand{\unun}{\textunderscore\ungap\textunderscore}
\newcommand{\xname}[1]{\unun\ungap#1}
\newcommand{\mname}[1]{\tcode{\unun\ungap#1\ungap\unun}}

%% Ranges
\newcommand{\Range}[4]{\tcode{#1#3,~\brk{}#4#2}\xspace}
\newcommand{\crange}[2]{\Range{[}{]}{#1}{#2}}
\newcommand{\brange}[2]{\Range{(}{]}{#1}{#2}}
\newcommand{\orange}[2]{\Range{(}{)}{#1}{#2}}
\newcommand{\range}[2]{\Range{[}{)}{#1}{#2}}

%% Change descriptions
\newcommand{\diffdef}[1]{\hfill\break\textbf{#1:}\xspace}
\newcommand{\change}{\diffdef{Change}}
\newcommand{\rationale}{\diffdef{Rationale}}
\newcommand{\effect}{\diffdef{Effect on original feature}}
\newcommand{\difficulty}{\diffdef{Difficulty of converting}}
\newcommand{\howwide}{\diffdef{How widely used}}

%% Miscellaneous
\newcommand{\uniquens}{\textrm{\textit{\textbf{unique}}}}
\newcommand{\stage}[1]{\item{\textbf{Stage #1:}}\xspace}
\newcommand{\doccite}[1]{\textit{#1}\xspace}
\newcommand{\cvqual}[1]{\textit{#1}}
\newcommand{\cv}{\cvqual{cv}}
\renewcommand{\emph}[1]{\textit{#1}\xspace}
\newcommand{\numconst}[1]{\textsl{#1}\xspace}
\newcommand{\logop}[1]{{\footnotesize #1}\xspace}

%%--------------------------------------------------
%% Environments for code listings.

% We use the 'listings' package, with some small customizations.  The
% most interesting customization: all TeX commands are available
% within comments.  Comments are set in italics, keywords and strings
% don't get special treatment.

\lstset{language=C++,
        basicstyle=\small\CodeStyle,
        keywordstyle=,
        stringstyle=,
        xleftmargin=1em,
        showstringspaces=false,
        commentstyle=\itshape\rmfamily,
        columns=flexible,
        keepspaces=true,
        texcl=true}

% Our usual abbreviation for 'listings'.  Comments are in 
% italics.  Arbitrary TeX commands can be used if they're 
% surrounded by @ signs.
\newcommand{\CodeBlockSetup}{
 \lstset{escapechar=@}
 \renewcommand{\tcode}[1]{\textup{\CodeStylex{##1}}}
 \renewcommand{\techterm}[1]{\textit{\CodeStylex{##1}}}
 \renewcommand{\term}[1]{\textit{##1}}
 \renewcommand{\grammarterm}[1]{\textit{##1}}
}

\lstnewenvironment{codeblock}{\CodeBlockSetup}{}
\lstnewenvironment{Program}{}{}

% A code block in which single-quotes are digit separators
% rather than character literals.
\lstnewenvironment{codeblockdigitsep}{
 \CodeBlockSetup
 \lstset{deletestring=[b]{'}}
}{}

% Permit use of '@' inside codeblock blocks (don't ask)
\makeatletter
\newcommand{\atsign}{@}
\makeatother

%%--------------------------------------------------
%% Indented text
\newenvironment{indented}
{\list{}{}\item\relax}
{\endlist}

%%--------------------------------------------------
%% Library item descriptions
\lstnewenvironment{itemdecl}
{
 \lstset{escapechar=@,
 xleftmargin=0em,
 aboveskip=2ex,
 belowskip=0ex	% leave this alone: it keeps these things out of the
				% footnote area
 }
}
{
}

\newenvironment{itemdescr}
{
 \begin{indented}}
{
 \end{indented}
}


%%--------------------------------------------------
%% Bnf environments
\newlength{\BnfIndent}
\setlength{\BnfIndent}{\leftmargini}
\newlength{\BnfInc}
\setlength{\BnfInc}{\BnfIndent}
\newlength{\BnfRest}
\setlength{\BnfRest}{2\BnfIndent}
\newcommand{\BnfNontermshape}{\small\rmfamily\itshape}
\newcommand{\BnfTermshape}{\small\ttfamily\upshape}
\newcommand{\nonterminal}[1]{{\BnfNontermshape #1}}

\newenvironment{bnfbase}
 {
 \newcommand{\nontermdef}[1]{\nonterminal{##1}\indexgrammar{\idxgram{##1}}:}
 \newcommand{\terminal}[1]{{\BnfTermshape ##1}\xspace}
 \newcommand{\descr}[1]{\normalfont{##1}}
 \newcommand{\bnfindentfirst}{\BnfIndent}
 \newcommand{\bnfindentinc}{\BnfInc}
 \newcommand{\bnfindentrest}{\BnfRest}
 \begin{minipage}{.95\hsize}
 \newcommand{\br}{\hfill\\}
 \frenchspacing
 }
 {
 \nonfrenchspacing
 \end{minipage}
 }

\newenvironment{BnfTabBase}[1]
{
 \begin{bnfbase}
 #1
 \begin{indented}
 \begin{tabbing}
 \hspace*{\bnfindentfirst}\=\hspace{\bnfindentinc}\=\hspace{.6in}\=\hspace{.6in}\=\hspace{.6in}\=\hspace{.6in}\=\hspace{.6in}\=\hspace{.6in}\=\hspace{.6in}\=\hspace{.6in}\=\hspace{.6in}\=\hspace{.6in}\=\kill}
{
 \end{tabbing}
 \end{indented}
 \end{bnfbase}
}

\newenvironment{bnfkeywordtab}
{
 \begin{BnfTabBase}{\BnfTermshape}
}
{
 \end{BnfTabBase}
}

\newenvironment{bnftab}
{
 \begin{BnfTabBase}{\BnfNontermshape}
}
{
 \end{BnfTabBase}
}

\newenvironment{simplebnf}
{
 \begin{bnfbase}
 \BnfNontermshape
 \begin{indented}
}
{
 \end{indented}
 \end{bnfbase}
}

\newenvironment{bnf}
{
 \begin{bnfbase}
 \list{}
	{
	\setlength{\leftmargin}{\bnfindentrest}
	\setlength{\listparindent}{-\bnfindentinc}
	\setlength{\itemindent}{\listparindent}
	}
 \BnfNontermshape
 \item\relax
}
{
 \endlist
 \end{bnfbase}
}

% non-copied versions of bnf environments
\newenvironment{ncbnftab}
{
 \begin{bnftab}
}
{
 \end{bnftab}
}

\newenvironment{ncsimplebnf}
{
 \begin{simplebnf}
}
{
 \end{simplebnf}
}

\newenvironment{ncbnf}
{
 \begin{bnf}
}
{
 \end{bnf}
}

%%--------------------------------------------------
%% Drawing environment
%
% usage: \begin{drawing}{UNITLENGTH}{WIDTH}{HEIGHT}{CAPTION}
\newenvironment{drawing}[4]
{
\newcommand{\mycaption}{#4}
\begin{figure}[h]
\setlength{\unitlength}{#1}
\begin{center}
\begin{picture}(#2,#3)\thicklines
}
{
\end{picture}
\end{center}
\caption{\mycaption}
\end{figure}
}

%%--------------------------------------------------
%% Environment for imported graphics
% usage: \begin{importgraphic}{CAPTION}{TAG}{FILE}

\newenvironment{importgraphic}[3]
{%
\newcommand{\cptn}{#1}
\newcommand{\lbl}{#2}
\begin{figure}[htp]\centering%
\includegraphics[scale=.35]{#3}
}
{
\caption{\cptn}\label{\lbl}%
\end{figure}}

%% enumeration display overrides
% enumerate with lowercase letters
\newenvironment{enumeratea}
{
 \renewcommand{\labelenumi}{\alph{enumi})}
 \begin{enumerate}
}
{
 \end{enumerate}
}

% enumerate with arabic numbers
\newenvironment{enumeraten}
{
 \renewcommand{\labelenumi}{\arabic{enumi})}
 \begin{enumerate}
}
{
 \end{enumerate}
}

%%--------------------------------------------------
%% Definitions section
% usage: \definition{name}{xref}
%\newcommand{\definition}[2]{\rSec2[#2]{#1}}
% for ISO format, use:
\newcommand{\definition}[2]{%
\subsection[#1]{\hfill[#2]}\vspace{-.3\onelineskip}\label{#2}\textbf{#1}\\%
}
\newcommand{\definitionx}[2]{%
\subsubsection[#1]{\hfill[#2]}\vspace{-.3\onelineskip}\label{#2}\textbf{#1}\\%
}
\newcommand{\defncontext}[1]{\textlangle#1\textrangle}

\NewEnviron{before}{}
%\newenvironment{before}{
%\begin{tcolorbox}[colback=red!5!white,colframe=red!75!black,title=Before]
%}{
%\end{tcolorbox}
%}

\newenvironment{after}{%
%\begin{tcolorbox}[colback=green!5!white,colframe=green!75!black,title=After]
}{%
%\end{tcolorbox}
}

% Wording on which the authors disagree.
\newenvironment{undecided}{}{}

\input{tables}

\makeindex[generalindex]
\makeindex[libraryindex]
\makeindex[impldefindex]

%%--------------------------------------------------
%% macros specific to the Networking TS
\newcommand{\completionsig}{\Fundesc{Completion signature}}
\newcommand{\DEDUCED}{\textit{\texttt{DEDUCED}}}
\newcommand{\nativeref}{see~\ref{socket.reqmts.native}}
\newcommand{\CppXref}[1]{\texorpdfstring{C\kern-0.05em\protect\raisebox{.35ex}{\textsmaller[2]{+\kern-0.05em+}}Std}{C++Std} [#1]}
% Alternative formatting of cross-references, resolving stable name to number.
% Needs include{cxx} and CppXIV from cplusplus/draft/source/macros.tex.
% \input{cxx}
% \newcommand{\CppXref}[1]{\CppXIV \S\cxxref{#1}}

% Add two names to the library index as both #1!#2 and #2!#1
\newcommand{\indexlibrarytwo}[2]{%
\indexlibrary{\idxcode{#1}!\idxcode{#2}}%
\indexlibrary{\idxcode{#2}!\idxcode{#1}}%
}
% Add an extensible implementation entry to the main and library name indices
\newcommand{\indexextensible}[1]{%
\indextext{extensible implementation!\idxcode{#1}}%
\indexlibrary{\idxcode{#1}!extensible implementation}%
}

%%--------------------------------------------------
%% fix interaction between hyperref and other
%% commands
\pdfstringdefDisableCommands{\def\smaller#1{#1}}
\pdfstringdefDisableCommands{\def\textbf#1{#1}}
\pdfstringdefDisableCommands{\def\raisebox#1{}}
\pdfstringdefDisableCommands{\def\hspace#1{}}

%%--------------------------------------------------
%% add special hyphenation rules
\hyphenation{tem-plate ex-am-ple in-put-it-er-a-tor name-space name-spaces}

%%--------------------------------------------------
%% turn off all ligatures inside \texttt
\DisableLigatures{encoding = T1, family = tt*}

% \errorcontextlines 10000

\begin{document}
\chapterstyle{cppstd}
\pagestyle{cpppage}

%%--------------------------------------------------
%% configuration
\input{config}

%%--------------------------------------------------
%% front matter
\frontmatter
\include{front}

%%--------------------------------------------------
%% main body of the document
\mainmatter
\setglobalstyles

\part[Merging Atom into Modules TS]{Merging Atom into the Modules TS}\label{atom}

This section presents the changes to be applied to the Modules TS in order
to merge the features of the Atom proposal, as directed by Evolution at the
Jacksonville 2018 committee meeting.

A note aboute stylistic conventions: as this document describes a ``diff of
a diff,'' the usual convention of using text style for \added{added} and
\removed{removed} text does not work well. In its place, we use block-style
diffs showing the text of the Modules TS before and after this wording is
applied:

\begin{before}
Here is some text from the \Cpp standard
\added{with some additions from the Modules TS}.
\end{before}

\begin{after}
Here is some text from the \Cpp standard
\added{with some additions from the Modules TS}
\added{and some more from the Atom proposal}.
\end{after}

Unchanged text from the Modules TS is retained in this section so that a
complete picture of the ``after'' wording may be obtained by simply ignoring
the ``before'' regions.
\newpage

\include{scope}
\include{references}
\include{definitions}
\include{general}
%!TEX root = ts.tex

\setcounter{chapter}{4}
\rSec0[lex]{Lexical conventions}

\setcounter{section}{0}
\Sec1[lex.separate]{Separate translation}

Modify paragraph 5.1/2 as follows
\begin{std.txt}
    \resetalinea[1]
    \alinea
    \enternote
    Previously translated translation units and instantiation units can be 
    preserved individually or in libraries. The separate translation units of 
    a program communicate (\ref{basic.link}) by (for example) calls to functions whose 
    identifiers have external \added{or module} linkage, manipulation of objects whose 
    identifiers have external \added{or module} linkage, or manipulation of data files. 
    Translation units can be separately translated and then later linked to produce 
    an executable program (\ref{basic.link}).  
    \exitnote

\end{std.txt}


\Sec1[lex.phases]{Phases of translation}

\noindent
Modify bullet 7 of paragraph 5.2/1 as follows:
\begin{std.txt}
%    \resetalinea[6]
%    \alinea
\begin{itemize}
    \item[7.]
    White-space characters separating tokens are no longer significant. 
    Each preprocessing token is converted into a token (5.6). 
    The resulting tokens are syntactically and semantically analyzed and 
    translated as a translation unit.
    \enternote
    The process of analyzing and translating the tokens may occasionally 
    result in one token being replaced by a sequence of other tokens (17.2). 
    \exitnote
    \added{It is implementation-defined whether the sources for
    module units and header units
    on which the current translation unit has an interface
    dependency (\ref{module.unit}, \ref{module.import})
    are required to be available.}
    \enternote
    Source files, translation units and translated translation units need not
     necessarily be stored as files, nor need there be any one-to-one 
     correspondence between these entities and any external representation. 
     The description is conceptual only, and does not specify any particular 
     implementation.
    \exitnote 
\end{itemize}
\end{std.txt}

\begin{after}
\setcounter{section}{3}
\Sec1[lex.pptoken]{Preprocessing tokens}

\textbf{\color{red}FIXME: The following change is a placeholder; the final rule for
context-sensitive lexing of \grammarterm{header-name} tokens
after \tcode{import} tokens is still under development.}

Modify bullet 3 of paragraph 5.4/3 as follows:

\begin{std.txt}
Otherwise, the next preprocessing token is
the longest sequence of characters
that could constitute a preprocessing token,
even if that would cause further lexical analysis to fail,
except that a \grammarterm{header-name} (5.8)
is only formed
\begin{itemize}
\item
within a \tcode{\#include} directive (19.2)\added{,}

\item
\added{within a \grammarterm{has-include-expression}, or}

\item
\added{outside of any preprocessing directive,
if applying phase 4 of translation to the sequence
of preprocessing tokens produced thus far
is valid and
results in an \grammarterm{import-seq} (\ref{cpp.module}).}
\end{itemize}
\end{std.txt}
\end{after}

\setcounter{section}{9}
\Sec1[lex.name]{Identifiers}

In \ref{lex.name}, add these two identifiers to Table 4,
``Identifiers with special meaning'':

\begin{std.txt}
\added{\tcode{module}}\\
\added{\tcode{import}}
\end{std.txt}

\setcounter{section}{10}
\Sec1[lex.key]{Keywords}

\noindent
Modify note in paragraph \ref{lex.key}/1 as follows:
\begin{std.txt}
    \resetalinea[0]
    \alinea
    ...


    \enternote
    The \removed{\tcode{export} and} \tcode{register} keyword\removed{s are} \added{is}
    unused but \removed{are} \added{is} 
    reserved for future use.
    \exitnote
\end{std.txt}

%!TEX root = std.tex
\setcounter{chapter}{5}
\rSec0[basic]{Basic concepts}

\rSec1[basic.def]{Declarations and definitions}

Modify paragraph 6.1/1 as follows:

\begin{std.txt}
  \pnum[1]
  A declaration (Clause \ref{dcl.dcl}) may introduce one or more names into a
  translation unit or redeclare names introduced by previous declarations.
  If so, the declaration specifies the interpretation and
  \removed{attributes}\added{semantic properties} of these names.
  [...]
  \end{std.txt}

\rSec1[basic.def.odr]{One-definition rule}

Change paragraph 6.2/1 as follows:
\begin{std.txt}
  \pnum[1]
  \removed{No translation unit shall contain more than one definition
  of any}
  \added{A}
  variable, function, class type, enumeration type, or template
  \added{shall not be defined where a prior definition is necessarily reachable
  (\ref{module.reach});
  no diagnostic is required if the prior declaration is in another translation unit}.
\end{std.txt}

\textbf{\color{red}FIXME: Should we disallow multiple definitions
of entities with module linkage now that we have module partitions?}

\begin{example}
\begin{codeblock}
// TU 1
module A;
class X {};
void f(X x) { /*...*/ }

// TU 2
module A;
class X {};
void f(X x);
void g() { X x; f(x); }
\end{codeblock}
\end{example}

Modify the end of paragraph 6.2/10 as follows

\begin{std.txt}
\pnum[10]
[...]
\removed{An}
\added{A definition of an}
inline function or variable shall be \removed{defined} \added{reachable}
in every translation unit in which it is odr-used outside of a discarded statement.
\end{std.txt}

Modify paragraph 6.2/11 as follows

\begin{std.txt}
\pnum[11]
\removed{Exactly one}
\added{A} definition of a class is required \added{to be reachable}
\removed{in a translation unit if} \added{in every context in which}
the class is used in a way that requires the class type to be complete.
\end{std.txt}

Modify opening of paragraph 6.2/12 as follows

\begin{std.txt}
  \pnum[12]
  There can be more than one definition of a class type (Clause 10), 
  enumeration type (9.6), inline function with external \added{or module} 
  linkage (9.1.6), 
  inline variable with external \added{or module} linkage (9.1.6), 
  class template (Clause 12), 
  non-static function template (12.6.6),
  static data member of a class 
  template (12.6.1.3), member function of a class template (12.6.1.1), or 
  template specialization for which some template parameters are not 
  specified (12.8, 12.6.5) in a program provided that
  \removed{each definition appears 
  in a different translation unit}
  \added{no prior definition is necessarily reachable (\ref{module.reach})
  at the point where a definition appears}, 
  and provided the definitions satisfy the 
  following requirements. 
\begin{before}
  \added{For an entity with an exported declaration, there shall be only one
   definition of that entity; a diagnostic is required only if the abstract 
   semantics graph of the module contains a definition of the entity.}
   \added{\enternote
   If the definition is not in the interface unit, then at most one module unit can
   have and make use of the definition.
   \exitnote}
\end{before}
\begin{after}
  \added{There shall not be more than one definition
  of an entity with external linkage
  that is attached to a named module (\ref{module.unit});
  no diagnostic is required unless a prior definition
  is reachable at a point where a later definition appears.}
\end{after}
  Given such an entity named \tcode{D} defined in more than one 
  translation unit, then
\end{std.txt}

\rSec1[basic.scope]{Scope}%

\setcounter{subsection}{5}
\rSec2[basic.scope.namespace]{Namespace scope}

Modify paragraph 6.3.6/1 as follows:
\begin{std.txt}
  \pnum[1]
  The declarative region of a \grammarterm{namespace-definition} is its
  \grammarterm{namespace-body}. 
  Entities declared in a \grammarterm{namespace-body}
  are said to be members of the namespace, and names introduced by
  these declarations 
  into the declarative region of the namespace are said to be
  \term{member names} of the namespace. A namespace member name
  has namespace
  scope. Its potential scope includes its namespace from the name's
  point of declaration (6.3.2) onwards;
   and for each \grammarterm{using-directive}
  (9.7.3) that nominates the member's namespace, the member's
  potential scope includes that portion of the potential scope of the
  \grammarterm{using-directive} that follows the member's point of
  declaration.
\begin{before}
\added{If a name $X$ (not having internal linkage) is declared
    in a namespace $N$ in the purview of the module
    interface unit of a module $M$, the potential scope of $X$ 
    includes the portion of the namespace $N$ in the purview of
     every module implementation unit of $M$
    and, if the name $X$ is exported, in every translation unit that imports $M$
    after a \grammarterm{module-import-declaration} nominating $M$.
    }
\end{before}
\begin{after}
\added{%
If a translation unit $M$ is imported into a translation unit $N$ (\ref{module.import}),
the potential scope of a name $X$ declared with namespace scope in $M$
is extended to include the portion of the corresponding namespace
scope in $N$ following the first \grammarterm{module-import-declaration}
or \grammarterm{module-declaration}
in $N$ that directly or indirectly imports $M$ if
} \color{addclr}
\begin{itemize}
\item \added{$X$ does not have internal linkage, and}
\item \added{$X$ is declared after the \grammarterm{module-declaration} in $M$ (if any), and}
\item \added{either $X$ is exported or $M$ and $N$ are part of the same module.}
\end{itemize}
\end{after}
  \color{addclr}
%  ; and for each \term{module-import-declaration} that nominates M,
%  the potential scope of X includes the 
%  portion of the potential scope of the \term{using-directive} that
%  follows that \term{module-import-declaration}. 
   \added{\enternote
    A \grammarterm{module-import-declaration} imports both the named
    translation unit(s) and any modules named by exported
    \grammarterm{module-import-declaration}{s} within them,
    recursively.
   \enterexample}
   \begin{codeblock}
    // Translation unit \#1
    export module M;
    export int sq(int i) { return i*i; }

    // Translation unit \#2
    export module N;
    export import M;

    // Translation unit \#3
    import N;
    int main() { return sq(9); }       // OK: \tcode{sq} from module \tcode{M}
    \end{codeblock}
    \added{
    \exitexample
   \exitnote}
\end{std.txt}


\setcounter{section}{3}
\rSec1[basic.lookup]{Name lookup}

Modify paragraph 6.4/1 as follows:
\begin{std.txt}
  \pnum[1]
  The name lookup rules apply uniformly to all names
  (including \grammarterm{typedef-name}{s} (9.1.3),
  \grammarterm{namespace-name}{s} (9.7), and 
  \grammarterm{class-name}{s} (10.1))
  wherever the grammar allows such names in the context discussed by a 
  particular rule.
  Name lookup associates the use of a name with a set of declarations (6.1)
  of that name.
  [...] Only after name lookup, function overload resolution (if applicable)
  and access checking have succeeded are the
  \removed{attributes}\added{semantic properties} introduced by
  the name's declaration
  \added{and its reachable (\ref{module.reach}) redeclarations}
  used further in the expression processing (Clause 7).
\end{std.txt}

% Moved to module.reach with some changes.
\begin{before}
Add new paragraph 6.4/5 as follows:
\begin{std.txt}\color{addclr}
  \pnum[5]
  \added{A declaration is \term{reachable} from a program point if it can be found
  by unqualified name lookup in its scope.}
\end{std.txt}
\end{before}

\setcounter{subsection}{1}
\rSec2[basic.lookup.argdep]{Argument-dependent name lookup}

Modify paragraph 6.4.2/2 as follows:
\begin{std.txt}
  \pnum[2]
  For each argument type \tcode{T} in the function call, there is a
  set of zero or more \term{associated namespaces} and a set of
  zero or more \term{associated \removed{classes} \added{entities}}
  \added{(other than namespaces)} to be considered.  The sets of
  namespaces and \removed{classes} \added{entities} are determined
  entirely by the types of the function arguments (and the namespace
  of any template template argument).  Typedef names and
  \grammarterm{using-declaration}{s} used to specify the types do not
  contribute to this set.  The sets of namespaces
  and \removed{classes} \added{entities} are determined in the
  following way:
  \begin{itemize}
  \item[---] If \tcode{T} is a fundamental type, its associated sets
  of namespaces and \removed{classes} \added{entities} are both empty.

  \item[---] If \tcode{T} is a class type (including unions), its
  associated \removed{classes} \added{entities} are the class itself;
  the class of which it is a member, if any; and its direct and
  indirect base classes.  Its associated namespaces are the innermost
  enclosing namespaces of its associated \removed{classes} \added{entities}.
  Furthermore, if \tcode{T} is a class template specialization, its
  associated namespaces and \removed{classes} \added{entities} also
  include: the namespace and \removed{classes} \added{entities}
  associated with the types of the template arguments provided for
  template type parameters (excluding template template arguments);
  \added{the templates used as template template arguments;} the
  namespaces of which any template template arguments are members;
  and the classes of which any member template used as template
  template arguments are members. \enternote Non-type template
  arguments do not contribute to the set of associated namespaces. \exitnote

  \item[---] If \tcode{T} is an enumeration type, its associated
  namespace is the innermost enclosing namespace of its
  declaration\added{, and its associated entities are \tcode{T} and,
  if}\removed{. If} it is a class member, \removed{its associated
  class is} the member's class\removed{; else it has no associated class}.

  \item[---] If \tcode{T} is a pointer to \tcode{U} or an array
  of \tcode{U}, its associated namespaces
  and \removed{classes} \added{entities} are those associated
  with \tcode{U}. 

  \item[---] If \tcode{T} is a function type, its associated
  namespaces and \removed{classes} \added{entities} are those
  associated with the function parameter types and those associated
  with the return type.

  \item[---] If \tcode{T} is a pointer to a data member of
  class \tcode{X}, its associated namespaces
  and \removed{classes} \added{entities} are those associated with the
  member type together with those associated with \tcode{X}.
  \end{itemize}

  If an associated namespace is an inline namespace (9.7.1), its
  enclosing namespace is also included in the set.  If an associated
  namespace directly contains inline namespaces, those inline
  namespaces are also included in the set.  In addition, if the
  argument is the name or address of a set of overloaded functions
  and/or function templates, its
  associated \removed{classes} \added{entities} and namespaces are the
  union of those associated with each of the members of the set, i.e.,
  the \removed{classes} \added{entities} and namespaces associated
  with its parameter types and return type.  Additionally, if the
  aforementioned set of overloaded functions is named with
  a \grammarterm{template-id}, its
  associated \removed{classes} \added{entities} and namespaces also
  include those of its type \grammarterm{template-argument}{s} and its
  template \grammarterm{template-argument}{s}.
\end{std.txt}

Modify paragraph 6.4.2/4 as follows:
\begin{std.txt}
  \pnum[4]
  When considering an associated namespace, the lookup is the same as
  the lookup performed when the associated namespace is used as a
  qualifier (6.4.3.2) except that:
  \begin{itemize}
    \item[---] Any \grammarterm{using-directive}{s} in the associated
  namespace are ignored.

     \item Any namespace-scope friend declaration functions or friend
  function templates (10.7.3) declared in \removed{associated}
  classes \added{in the set of associated entities} are visible within
  their respective namespaces even if they are not visible during an
  ordinary lookup (9.7.1.2).

     \item All names except those of (possibly overloaded) functions
  and function templates are ignored.

     \color{addclr}
     \item[---]
     \added{Any function or function template in the interface of
     a named module \tcode{M} (\ref{module.interface}) that has
     the same innermost enclosing non-inline namespace as
     some associated entity attached to \tcode{M}
     is visible within its namespace
     to a lookup that does not occur within a module unit of \tcode{M},
     even if it is not visible during an ordinary lookup.}

    \item[---]
    \added{All declarations that are visible at any point in the instantiation
    context (\ref{module.context}) of the lookup are visible
    even if they are not visible during an ordinary lookup,
    excluding discarded declarations (\ref{module.global}) and
    internal linkage declarations attached to the global module.}
  \end{itemize}

\textbf{\color{red}FIXME: Add an example.}
\end{std.txt}

\setcounter{section}{4}
\rSec1[basic.link]{Program and linkage}%

Change the definition of \term{translation-unit} in paragraph 6.5/1
to:

\begin{bnf}
\nonterminal{translation-unit}:\br
    \added{top-level-}declaration-seq\opt\br
    \added{global-module-fragment\opt module-declaration top-level-declaration-seq\opt private-module-fragment\opt}
\end{bnf}

\begin{bnf}
\added{\nonterminal{private-module-fragment}:\br
    \terminal{module} \terminal{:} \terminal{private} \terminal{;} top-level-declaration-seq\opt}
\end{bnf}

\begin{bnf}\color{addclr}
      \added{\nonterminal{top-level-declaration-seq}:\br
          top-level-declaration\br
          top-level-declaration-seq top-level-declaration}
\end{bnf}

\begin{bnf}\color{addclr}
      \added{\nonterminal{top-level-declaration}:\br
         module-import-declaration\br
         declaration}
\end{bnf}

Add new paragraphs after the grammar:

\begin{std.txt}
\pnum[1]
\added{A \grammarterm{private-module-fragment} shall appear only
in a primary module interface unit (\ref{module.unit}).}

\pnum
\added{A token sequence beginning with
\tcode{export}\opt{} \tcode{module} or
\tcode{export}\opt{} \tcode{import}
and not immediately followed by \tcode{::}
is never interpreted as the \grammarterm{declaration}
of a \grammarterm{top-level-declaration}.}
\end{std.txt}

Insert a new bullet between first and second bullet of paragraph 6.5/2:

\begin{std.txt}\color{addclr}
  \begin{itemize}
    \item[---] \added{When a name has \term{module linkage}, the entity it denotes
      can be referred to by names
      from other scopes of the same module unit (\ref{module.unit}) 
      or from scopes of
      other module units of that same module.}
  \end{itemize}
\end{std.txt}

Modify bullet (3.2) of paragraph 6.5/3 as follows:
\begin{std.txt}
  \begin{itemize}
    \item[---]
    %a non-inline \added{non-exported} variable of non-volatile const-qualified type 
    %that is neither explicitly declared \tcode{extern} nor
    %previously declared to have external \added{or module} linkage; or
    a \removed{non-inline} variable of non-volatile const-qualified type\added{,
    unless}
    \begin{itemize}
      \color{addclr}
      \item[---] \added{it is explicitly declared \tcode{extern}, or}
      \item[---] \added{it is inline or exported, or}
      \item[---] \added{it was previously declared and the prior declaration did
      not have internal linkage; or}
    \end{itemize}
    \removed{that is neither explicitly declared \tcode{extern} nor
    previously declared to have external linkage; or}
  \end{itemize}
\end{std.txt}

Modify paragraph 6.5/4 as follows:

\begin{std.txt}
  \pnum[4]
  An unnamed namespace or a namespace declared directly or
  indirectly within an unnamed namespace has internal linkage. All other
  namespaces have external linkage.
  A name having namespace scope that has
  not been given internal linkage above \removed{has the same linkage as the enclosing
  namespace if it}\added{and that} is the name of
  \begin{itemize}
    \item a variable; or
    \item a function; or
    \item a named class (Clause 10), or an unnamed class defined in a
    typedef declaration in which the class has the typedef name for
    linkage purposes (9.1.3); or
    \item a named enumeration (9.6), or an unnamed enumeration defined in
    a typedef declaration in which the enumeration has the typedef name for
    linkage purposes (9.1.3); or
    \item a template\removed{.}
  \end{itemize}
  \begin{before}\color{addclr}
  \added{has the same linkage as the enclosing namespace if}
    \begin{itemize}
      \item \added{said namespace has internal linkage, or}
      \item \added{the name is exported (\ref{module.interface}), or}
      \added{is declared in a \grammarterm{proclaimed-ownership-declaration},
      or is not being declared in the purview of
      a named module (\ref{module.unit});}
    \end{itemize}
    \added{otherwise, the name has module linkage.}
  \end{before}
  \begin{after}\color{addclr}
  \added{has its linkage determined as follows:}
    \begin{itemize}\color{addclr}
      \item \added{if the enclosing namespace has internal linkage,
      the name has internal linkage;}
      \item \added{otherwise, if
      the declaration of the name is
      attached to a named module (\ref{module.unit})
      and is not exported (\ref{module.interface}),
      the name has module linkage;}
      \item \added{otherwise, the name has external linkage}.
    \end{itemize}
  \end{after}
\end{std.txt}

Modify 6.5/6 as follows:
\begin{std.txt}
\pnum[6]
The name of a function declared in block scope and the name of a
variable declared by a block scope extern declaration have linkage.
\added{If such a declaration is attached to a named module,
the program is ill-formed.}
If there is a visible declaration of an entity with linkage having the
same name and type, ignoring entities declared outside the innermost
enclosing namespace scope, the block scope declaration declares that
same entity and receives the linkage of the previous declaration. 
If
there is more than one such matching entity, the program is
ill-formed. Otherwise, if no matching entity is found, the block scope
entity receives external linkage.
\end{std.txt}

Modify paragraph 6.5/10 and add /11 as follows:
\begin{std.txt}
  \pnum[10]
  \addtocounter{footnote}{1}
  Two
  names that are the same (Clause 6) and that are declared in different scopes
  shall denote the same variable, function, type, template or namespace if
  \begin{itemize}
    \item[---] both names have external \added{or module} linkage
      \added{and are declared in declarations attached to the
      same module,}
      or else both names have internal linkage and are declared in
      the same translation unit; and

      \item both names refer to members of the same namespace or to members,
      not by inheritance, of the same class; and

      \item when both names denote functions, the parameter-type-lists of the
      functions (9.2.3.5) are identical; and

      \item when both names denote function templates, the signatures (12.6.6.1)
      are the same.
  \end{itemize}
  \added{If multiple declarations of the same name with external linkage
  would declare the same entity except that they are attached to different
  modules, the program is ill-formed; no diagnostic is required.
  \enternote
  \grammarterm{using-declaration}{s}, typedef declarations, and 
  \grammarterm{alias-declaration}{s} do not declare entities, but 
  merely introduce synonyms. Similarly, \grammarterm{using-directive}{s}
  do not declare entities.
  \exitnote}

  \pnum
  \added{If a declaration would redeclare a reachable declaration
  attached to a different module, the program is ill-formed.}
  \color{addclr}
  \begin{example}
    \begin{codeblock}
      // \tcode{"decls.h"}
      int f();            // \#1, attached to the global module
      int g();            // \#2, attached to the global module

      // module interface of \tcode{M}
      module;
      #include "decls.h"
      export module M;
      export using ::f;   // OK: does not declare an entity, exports \#1
      int g();            // error: matches \#2, but attached to M
      export int h();     // \#3
      export int k();     // \#4

      // other translation unit
      import M;
      static int h();     // error: matches \#3
      int k();            // error: matches \#4
    \end{codeblock}
  \end{example}
  \added{As a consequence of these rules,
  all declarations of an entity are \term{attached} to the same module;
  the entity is said to be attached to that module.}
\end{std.txt}


\setcounter{section}{7}
\rSec1[basic.exec]{Program execution}

\setcounter{subsection}{2}
\rSec2[basic.start]{Start and termination}

\setcounter{subsubsection}{0}
\rSec3[basic.start.main]{\tcode{main} function}

Modify paragraph 6.8.3.1/1 as follows:
\begin{std.txt}
  \pnum[1]
  A program shall contain a global function called \tcode{main}
  \added{attached to the global module}.
\end{std.txt}

Modify paragraph 6.8.3.1/3 as follows:
\begin{std.txt}
  \pnum[3]
  ...
  A program that declares a variable \tcode{main} at global scope\added{,
  or that declares a function \tcode{main} at global scope attached to a named module,}
  or that declares the name \tcode{main} with C language linkage (in any namespace)
  is ill-formed.
\end{std.txt}

%!TEX root = std.tex
\setcounter{chapter}{8}
\rSec0[dcl.dcl]{Declarations}%
%\indextext{declaration|(}

%gram: \rSec1[gram.dcl]{Declarations}
%gram:

\noindent
Add new alternatives to \term{declaration} in paragraph 9/1 as follows
\begin{std.txt}
  \begin{bnf}
    \nonterminal{declaration}:\br
      block-declaration\br
      nodeclspec-function-declaration\br
      function-definition\br
      template-declaration\br
      explicit-instantiation\br
      explicit-specialization\br
      linkage-specification\br
      namespace-definition\br
      empty-declaration\br
      attribute-declaration\br
      \added{export-declaration}\br
  \end{bnf}
  \end{std.txt}


\setcounter{section}{0}
\rSec1[dcl.spec]{Specifiers}%

\setcounter{subsection}{5}
\rSec2[dcl.inline]{The \tcode{inline} specifier}%

\noindent
Modify paragraph 9.1.6/6 as follows
\begin{std.txt}
  \pnum[6]
  \removed{An}
  \added{If an}
  inline function or variable
  \added{is odr-used in a}
  \removed{shall be defined in every}
  translation unit\added{,
  a definition of it shall be reachable from the end of that translation unit,}
  \removed{in which it is odr-used} and \added{it} shall have exactly 
  the same definition in every \added{such translation unit} \removed{case} (\ref{basic.link}).
  \enternote
  A call to the inline function or a use of 
  the inline variable may be encountered before its definition appears 
  in the translation unit.
  \exitnote
  % 
  If \removed{the} \added{a} definition of a function or variable
  \removed{appears in a translation unit before}
  \added{is reachable at the point of}
  its first declaration as inline,
  the program is ill-formed. If a 
  function or variable with external \added{or module} linkage is 
  declared inline in one translation 
  unit,
  \added{there} \removed{it} shall be
  \added{a reachable} \removed{declared} inline
  \added{declaration} in all translation units in which it 
  \removed{appears}\added{is declared}; 
  no diagnostic is required. An inline function or variable with 
  external \added{or module} linkage 
  shall have the same address in all translation units.
  \enternote
  A \tcode{static} local variable in an inline function with external 
  \added{or module} linkage always refers to the same object. A type 
  defined within the body 
  of an inline function with external \added{or module} linkage is the 
  same type in every translation unit.
  \exitnote
\end{std.txt}

\noindent
Add a new paragraph 9.1.6/7 as follows:
\begin{std.txt}
  \color{addclr}
  \pnum[7]
  \added{An exported inline function or variable
  shall be defined in the translation unit
  containing its exported declaration,
  outside the \grammarterm{private-module-fragment} (if any).
  \enternote
  There is no restriction on the linkage (or absence thereof)
  of entities that the function body of an exported inline function
  can reference. A constexpr function (9.1.5) is implicitly inline.
  \exitnote}
\end{std.txt}

\setcounter{subsection}{6}
\rSec2[dcl.type]{Type specifiers}%

\setcounter{subsubsection}{3}
\rSec3[dcl.spec.auto]{The \tcode{auto} specifier}%

\noindent
Add a new paragraph before 9.1.7.4/9 (``If the name of an entity with an
undeduced placeholder type appears in an expression, the program is
ill-formed.'') as follows:
\begin{std.txt}
\pnum[9]
\added{An exported function
with a declared return type that uses a placeholder type
shall be defined in the translation unit
containing its exported declaration,
outside the \grammarterm{private-module-fragment} (if any).
\enternote
There is no restriction on the linkage of
the deduced return type.
\exitnote}
\end{std.txt}

\setcounter{section}{6}
\rSec1[basic.namespace]{Namespaces}%
%\indextext{namespaces|(}

\noindent
Add a new paragraph after 9.7/1 as follows:
\begin{std.txt}
  \pnum[1]
  A namespace is an optionally-named declarative region. The name of a
  namespace can be used to access entities declared in that namespace;
  that is, the members of the namespace. Unlike other declarative
  regions, the definition of a namespace can be split over several
  parts of one or more translation units.

  \pnum
  \added{\enternote A namespace name with external linkage is exported
  if any of its \term{namespace-definition}{s} is
  exported, or if it contains any
  \grammarterm{export-declaration}{s} (\ref{module.interface}).
  A namespace is never attached to a module, and never has module
  linkage even if it is not exported.
  \exitnote}
  \color{addclr}
  \begin{example}
  \begin{codeblock}
    export module M;
    namespace N1 {}                     // \tcode{N1} is not exported
    export namespace N2 {}              // \tcode{N2} is exported
    namespace N3 { export int n; }      // \tcode{N3} is exported
  \end{codeblock}
  \end{example}
\end{std.txt}

%!TEX root = std.tex
\setcounter{chapter}{9}
\rSec0[class]{Classes}

\setcounter{section}{1}
\rSec1[class.mem]{Class members}

\setcounter{subsection}{9}
\rSec2[class.bit]{Bit-fields}

Modify paragraph 10.2.10/1 as follows:
\begin{std.txt}
    \pnum[1]
    [...] The bit-field \removed{attribute}\added{semantic property} is not part of 
    the type of the class member. [...]
\end{std.txt}

%!TEX root = ts.tex

\setcounter{chapter}{10}
\rSec0[over]{Overloading}

\setcounter{section}{4}
\rSec1[over.oper]{Overloaded operators}

\setcounter{subsection}{7}
\rSec2[over.literal]{User-defined literals}

\noindent
Modify paragraph 11.5.8/7 as follows:
\begin{std.txt}
    \pnum[7]
    \enternote
    Literal operators and literal operator templates are usually invoked 
    implicitly through user-defined literals (5.13.8). However, except for 
    the constraints described above, they are ordinary namespace-scope 
    functions and function templates. In particular, they are looked up 
    like ordinary functions and function templates and they follow the same 
    overload resolution rules. Also, they can be declared \tcode{inline}
     or \tcode{constexpr}, 
    they can have internal\added{, module,} or external linkage, they can be called explicitly, 
    their addresses can be taken, etc.
    \exitnote
\end{std.txt}

%!TEX root = std.tex
\setcounter{chapter}{11}
\rSec0[temp]{Templates}%
%\indextext{template|(}

\noindent
Modify paragraph 12/4 as follows:
\begin{std.txt}
  \pnum[4]
  A \grammarterm{template-declaration} can appear only as a 
  namespace scope or class scope declaration.
  \added{Its \grammarterm{declaration} shall not be an
  \grammarterm{export-declaration}.}
  In a function template declaration, the last component of the 
  \grammarterm{declarator-id} shall not be a \grammarterm{template-id}. 
  [...]
\end{std.txt}


%gram: \rSec1[gram.temp]{Templates}
%gram:

%% \setcounter{section}{5}
%% \rSec1[temp.res]{Name resolution}

%% \setcounter{subsection}{3}
%% \rSec2[temp.dep.res]{Dependent name resolution}

%% \noindent
%% Modify second bullet of paragraph 14.6.4/1
%% \begin{std.txt}
%%   \item[---] Declarations from namespace \added{partitions} associated
%%     with the types of the function arguments both from the
%%     instantiation context (14.6.4.1) and from the definition context. 
%% \end{std.txt}

\setcounter{section}{6}
\rSec1[temp.res]{Name resolution}
\setcounter{subsection}{3}
\rSec2[temp.dep.res]{Dependent name resolution}

Change in 12.7.4/1:

\begin{std.txt}
\pnum[1]
In resolving dependent names, names from the following sources are considered:
\begin{itemize}
\item
Declarations that are visible at the point of definition of the template.
\item
Declarations from namespaces associated with the types of the function
arguments both from the instantiation context
\removed{(12.7.4.1)} \added{(\ref{module.context})}
and from the definition context.
\end{itemize}
\enterexample\color{addclr}
\begin{codeblock}
// header file \tcode{"X.h"}
namespace Q { 
  struct X { };
}

// header file \tcode{"G.h"}
namespace Q {
  void g_impl(X, X);
}

// interface unit of \tcode{M1}
module;
#include "X.h"
#include "G.h"
export module M1;
export template<typename T>
void g(T t) {
  g_impl(t, Q::X{ });   // ADL in definition context finds \tcode{Q::g_impl}, \tcode{g_impl} not discarded
}

// interface unit of \tcode{M2}
module;
#include "X.h"
export module M2;
import M1;
void h(Q::X x) {
   g(x);                // OK
}
\end{codeblock}
\exitexample
\end{std.txt}

\noindent
Add new paragraphs to 12.7.4:
\begin{std.txt}
\color{addclr}
\pnum
\enterexample
\begin{codeblock}
// interface unit of \tcode{Std}
export module Std;
export template<typename Iter>
void indirect_swap(Iter lhs, Iter rhs)
{
  swap(*lhs, *rhs);     // swap can be found only via ADL
}

// interface unit of \tcode{M}
module;
import Std;
export module M;

struct S { /* ...*/ };
void swap(S&, S&);      // \#1;

void f(S* p, S* q)
{
  indirect_swap(p, q);  // finds \#1 via ADL in instantiation context
}
\end{codeblock}
\exitexample

\pnum
\enterexample
\begin{codeblock}
// header file \tcode{"X.h"}
struct X { /* ... */ };
X operator+(X, X);

// module interface unit of \tcode{F}
export module F;
export template<typename T>
void f(T t) {
  t + t;
}

// module interface unit of \tcode{M}
module;
#include "X.h"
import F;
export module M;
void g(X x) {
  f(x);             // OK: instantiates \tcode{f} from \tcode{F},
                    // \tcode{operator+} is visible in instantiation context
}
\end{codeblock}
\exitexample

\pnum
\enterexample
\begin{codeblock}
// module interface unit of \tcode{A}
export module A;
export template<typename T>
void f(T t) {
  t + t;           // \#1
}

// module interface unit of \tcode{B}
export module B;
import A;
export template<typename T, typename U>
void g(T t, U u) {
  f(t);
}

// module interface unit of \tcode{C1}
module;
#include <string>   // \tcode{operator+} not referenced, discarded
export module C1;
import B;
export template<typename T>
void h(T t) {
  g(std::string{ }, t);
}

// translation unit
import C1;
void i() {
   h(0);        // ill-formed: '+' not found at \#1
}

// module interface unit of \tcode{C2}
export module C2;
import B;
import <string>;
export template<typename T>
void j(T t) {
  g(std::string{ }, t);
}

// translation unit
import C2;
void k() {
   j(0);        // OK, '+' found in instantiation context:
                // visible at end of module interface unit of \tcode{C2}
}
\end{codeblock}
\exitexample
\end{std.txt}

\rSec3[temp.point]{Point of instantiation}

\noindent
Delete paragraph 12.7.4.1/7:
\begin{std.txt}
\pnum[7]
\removed{The instantiation context of an expression that depends on
the template arguments is the set of declarations with external
linkage declared prior to the point of instantiation of the
template specialization in the same translation unit.}
\end{std.txt}

\noindent
Change in paragraph \ref{temp.point}/8:
\begin{std.txt}
\pnum[8]
in addition to the points of instantiation described above, for any such
specialization that has a point of instantiation within the
\added{\grammarterm{declaration-seq} of the}
translation unit,
\added{prior to the \grammarterm{private-module-fragment} (if any),
the point after the \grammarterm{declaration-seq}
of the \grammarterm{translation-unit}
is also considered a point of instantiation,
and for any such specialization that has a point of instantiation
within the \grammarterm{private-module-fragment},}
the end of the translation unit is also
considered a point of instantiation.
\end{std.txt}

\rSec3[temp.dep.candidate]{Candidate functions}

\noindent
Modify paragraph 12.7.4.2/1 as follows
\begin{std.txt}
\pnum[1]
\ldots
If the call would be ill-formed or would find a better match had the 
lookup within the associated namespaces considered all the function 
declarations with external   \added{or module}
linkage introduced in those namespaces in all
translation units, not just considering those declarations found in the 
template definition and template instantiation contexts, then the program 
has undefined behavior.
\end{std.txt}




%!TEX root = std.tex
\setcounter{chapter}{13}
\rSec0[cpp]{Preprocessing directives}%

\begin{after}
Modify paragraph 14/5 as follows:

\begin{std.txt}
\resetalinea[4]
\alinea
The implementation can
process and skip sections of source files conditionally,
include other source files,
\added{import macros from header units,}
and replace macros.
These capabilities are called
\term{preprocessing},
because conceptually they occur
before translation of the resulting translation unit.
\end{std.txt}
\end{after}

\setcounter{section}{0}
\rSec1[cpp.cont]{Conditional inclusion}%

Modify the grammar before 14.1/1 as follows:

\begin{std.txt}
\begin{bnf}
\added{\nonterminal{header-name-tokens}:}\br
  \added{string-literal}\br
  \added{\terminal{<} h-pp-tokens \terminal{>}}
\end{bnf}

\begin{bnf}
\nonterminal{has-include-expression}:\br
  % FIXME: Need \textunderscore: underscore package does not work inside \added / \removed!
  \removed{\terminal{\xname{has\textunderscore{}include}} \terminal{(} \terminal{<} h-char-sequence \terminal{>} \terminal{)}}\br
  \removed{\terminal{\xname{has\textunderscore{}include}} \terminal{(} \terminal{"} q-char-sequence \terminal{"} \terminal{)}}\br
  \added{\terminal{\xname{has\textunderscore{}include}} \terminal{(} header-name \terminal{)}}\br
  \removed{\terminal{\xname{has\textunderscore{}include}} \terminal{(} string-literal \terminal{)}}\br
  \removed{\terminal{\xname{has\textunderscore{}include}} \terminal{(} \terminal{<} h-pp-tokens \terminal{>} \terminal{)}}\br
  \added{\terminal{\xname{has\textunderscore{}include}} \terminal{(} header-name-tokens \terminal{)}}\br
\end{bnf}
\end{std.txt}

Modify paragraph 14.1/2 as follows:

\begin{std.txt}
\resetalinea[1]
\alinea
A \grammarterm{defined-macro-expression} evaluates to \tcode{1}
if the identifier is currently defined as a macro name
(that is, if it is predefined or if
\added{it has one or more active macro definitions (\ref{cpp.module}),
for example because}
it has been the subject of a \tcode{\#define} preprocessing directive
without an intervening \tcode{\#undef} directive with the same subject
identifier),
\tcode{0} if it is not.

\end{std.txt}

Modify paragraph 14.1/3 as follows:

\begin{std.txt}
\resetalinea[2]
\alinea
The \removed{third and fourth forms}
\added{second form}
of \grammarterm{has-include-expression}
\removed{are} \added{is} considered
only if \removed{neither of} the first \removed{or second forms matches}
\added{form does not match},
in which case the preprocessing tokens are processed just as in normal text.
\end{std.txt}

\setcounter{section}{1}
\rSec1[cpp.include]{Source file inclusion}%

\begin{after}
Add a new paragraph after 14.2/6 as follows:

\begin{std.txt}
\resetalinea[6]
\color{addclr}
\alinea
If the header identified by the \grammarterm{header-name}
denotes an importable header (\ref{module.import}),
the preprocessing directive
is instead replaced by the \grammarterm{preprocessing-token}{s}

\begin{bnf}
\terminal{import} header-name \terminal{;}
\end{bnf}
\end{std.txt}
\end{after}

\noindent
Add a new subclause 14.3 titled ``\textbf{Global module fragment}'' as follows:

\setcounter{section}{2}
\rSec1[cpp.glob.frag]{Global module fragment}%
\resetalinea[0]

\begin{std.txt}
\color{addclr}
\begin{bnf}
\nonterminal{pp-global-module-fragment}:\br
  \terminal{module} \terminal{;} pp-bracketed-tokens \terminal{module}
\end{bnf}

\alinea
If the first two preprocessing tokens at the start of phase 4 of translation
are \tcode{module} \tcode{;}, the result of preprocessing shall begin with
a \grammarterm{pp-global-module-fragment} for which all
\grammarterm{preprocessing-token}{s} in the \grammarterm{pp-bracketed-tokens}
were produced directly or indirectly by source file inclusion
(\ref{cpp.include}), and for which the second \tcode{module}
\grammarterm{preprocessing-token} was not produced by source file inclusion or
macro replacement (\stdref{cpp.replace}{14.3}).
Otherwise, the first two preprocessing tokens at the end of phase 4 of
translation shall not be \tcode{module} \tcode{;}.
\end{std.txt}

Add a new subclause 14.4 titled ``\textbf{Legacy header units}'' as follows:

\setcounter{section}{3}
\rSec1[cpp.module]{Legacy header units}%
\resetalinea[0]

\begin{std.txt}
\color{addclr}
\begin{bnf}
\nonterminal{import-seq}:\br
  \grammarterm{top-level-token-seq}\opt{} \terminal{export}\opt{} \terminal{import}
\end{bnf}

\begin{bnf}
\nonterminal{top-level-token-seq}:\br
  \descr{any \nonterminal{pp-balanced-token-seq} ending in \terminal{;} or \terminal{\}}}
\end{bnf}

\begin{bnf}
\nonterminal{pp-import}:\br
  \terminal{import} header-name pp-import-suffix\opt{} \terminal{;}\br
  \terminal{import} header-name-tokens pp-import-suffix\opt{} \terminal{;}
\end{bnf}

\begin{bnf}
\nonterminal{pp-import-suffix}:\br
  pp-import-suffix-token\br
  pp-import-suffix pp-import-suffix-token
\end{bnf}

\begin{bnf}
\nonterminal{pp-import-suffix-token}:\br
  \descr{any \grammarterm{pp-balanced-token} other than \terminal{;}}
\end{bnf}

\begin{bnf}
\nonterminal{pp-balanced-token-seq}:\br
  pp-balanced-token\br
  pp-balanced-token-seq pp-balanced-token
\end{bnf}

\begin{bnf}
\nonterminal{pp-balanced-token}:\br
  pp-ldelim pp-balanced-token-seq\opt{} pp-rdelim\br
  \descr{any \grammarterm{preprocessing-token} other than a \nonterminal{pp-ldelim} or \nonterminal{pp-rdelim}}
\end{bnf}

\begin{bnf}
\nonterminal{pp-ldelim:} \descr{one of}\br
  \terminal{(    [    \{    <:    <\%}
\end{bnf}

\begin{bnf}
\nonterminal{pp-rdelim:} \descr{one of}\br
  \terminal{)    ]    \}    :>    \%>}
\end{bnf}

\color{addclr}
\alinea
A sequence of \grammarterm{preprocessing-token}{s} matching the form
of a \grammarterm{pp-import}
instructs the preprocessor to import macros from the header unit
(\ref{module.import}) denoted by the \grammarterm{header-name}.
A \grammarterm{pp-import} is only recognized when the sequence of tokens
produced by phase 4 of translation up to the \tcode{import} token
forms a \grammarterm{import-seq}, and the \tcode{import} token is not
within the \grammarterm{pp-import-suffix} of another \grammarterm{pp-import}.
The \tcode{;} \grammarterm{preprocessing-token} shall not be produced by
macro replacement (\stdref{cpp.replace}{14.3}).
The \term{point of macro import} for a \grammarterm{pp-import} is
immediately after the \tcode{;} terminating the \grammarterm{pp-import}.

\color{addclr}
\alinea
In the second form of \grammarterm{pp-import},
a \grammarterm{header-name} token is formed as if
the \grammarterm{header-name-tokens}
were the \grammarterm{pp-tokens} of a \tcode{\#include} directive.
The \grammarterm{header-name-tokens} are replaced by
the \grammarterm{header-name} token.
\begin{note}
This ensures that imports are treated consistently by
the preprocessor and later phases of translation.
\end{note}

\color{addclr}
\alinea
Each \tcode{\#define} directive encountered when preprocessing
each translation unit in a program results in a distinct
\term{macro definition}.
Importing macros from a header unit makes macro definitions
from a translation unit visible in other translation units.
Each macro definition has at most one point of definition in
each translation unit and at most one point of undefinition, as follows:
\begin{itemize}
\item
The \term{point of definition} of a macro definition within a translation unit
is the point at which its \tcode{\#define} directive occurs (in the translation
unit containing the \tcode{\#define} directive), or,
if the macro name is not lexically identical to a keyword (\stdref{lex.key}{5.11})
or to the \grammarterm{identifier}{s} \tcode{module} or \tcode{import},
the first point
of macro import of a translation unit containing a point of definition for the
macro definition, if any (in any other translation unit).

\item
The \term{point of undefinition} of a macro definition within a translation unit
is the first point at which a \tcode{\#undef} directive naming the macro occurs
after its point of definition, or the first point
of macro import of a translation unit containing a point of undefinition for the
macro definition, whichever (if any) occurs first.
\end{itemize}

\alinea
A macro directive is \term{active} at a source location
if it has a point of definition in that translation unit preceding the location,
and does not have a point of undefinition in that translation unit preceding
the location.

\alinea
If a macro would be replaced or redefined, and multiple macro definitions
are active for that macro name, the active macro definitions shall all be
valid redefinitions of the same macro (\stdref{cpp.replace}{14.3}).
\enternote
The relative order of \grammarterm{pp-import}{s} has no bearing on whether a
particular macro definition is active.
\exitnote

\alinea
\begin{example}
\begin{codeblock}
// translation unit a.h
#define X 123 // \#1
#define Y 45  // \#2
#define Z a   // \#3
#undef X      // point of undefinition of \#1 in \tcode{a.h}
\end{codeblock}

\begin{codeblock}
// translation unit b.h
import "a.h"; // point of definition of \#1, \#2, and \#3, point of undefinition of \#1 in \tcode{b.h}
#define X 456 // OK, \#1 is not active
#define Y 6   // ill-formed; \#2 is active
\end{codeblock}

\begin{codeblock}
// translation unit c.h
#define Y 45  // \#4
#define Z c   // \#5
\end{codeblock}

\begin{codeblock}
// translation unit d.h
import "a.h"; // point of definition of \#1, \#2, and \#3, point of undefinition of \#1 in \tcode{d.h}
import "c.h"; // point of definition of \#4 and \#5 in \tcode{d.h}
int a = Y;    // OK, active macro definitions \#2 and \#3 are valid redefinitions
int c = Z;    // ill-formed; active macro definitions \#2 and \#3 are not valid redefinitions of \tcode{Z}
\end{codeblock}
\end{example}
\end{std.txt}


%%--------------------------------------------------
%% appendices

%\appendix
%\include{compatibility}

\makeatletter
\immediate\closeout\XTR@out
\immediate\openout\XTR@out=cxx20-dummy.tmp
\makeatother

\part[Merging Modules TS into C++20]{Merging parts of the Modules TS into C++20}\label{atom}

This section presents the changes to be applied to the C++20 working draft in
order to merge the features representing the common subset of version 1 of the
Modules TS and the Atom proposal.

\renewenvironment{before}{\nullfont}{}
\renewenvironment{after}{}{}

\newpage
\input{cxx20.ext}

%%--------------------------------------------------
%% back matter
%\backmatter
%\include{back}


%%--------------------------------------------------
%% End of document
\end{document}

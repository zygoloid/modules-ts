%% main file for the C++ standard.
%%

%%--------------------------------------------------
%% basics
\documentclass[ebook,10pt,oneside,openany,final]{memoir}

\usepackage[american]
           {babel}        % needed for iso dates
\usepackage[iso,american]
           {isodate}      % use iso format for dates
\usepackage[final]
           {listings}     % code listings
\usepackage{longtable}    % auto-breaking tables
\usepackage{ltcaption}    % fix captions for long tables
\usepackage{booktabs}     % fancy tables
\usepackage{relsize}      % provide relative font size changes
\usepackage{underscore}   % remove special status of '_' in ordinary text
\usepackage{verbatim}     % improved verbatim environment
\usepackage{parskip}      % handle non-indented paragraphs "properly"
\usepackage{array}        % new column definitions for tables
\usepackage[normalem]{ulem}
\usepackage{color}        % define colors for strikeouts and underlines
\usepackage{amsmath}      % additional math symbols
\usepackage{mathrsfs}     % mathscr font
\usepackage{microtype}
\usepackage{multicol}
\usepackage{xspace}
\usepackage{fixme}
\usepackage{lmodern}
\usepackage[T1]{fontenc}
\usepackage[pdftex, final]{graphicx}
\usepackage[pdftex,
            pdftitle={C++ Technical Specification},
            pdfsubject={C++ Technical Specification},
            pdfcreator={Gabriel Dos~Reis},
            bookmarks=true,
            bookmarksnumbered=true,
            pdfpagelabels=true,
            pdfpagemode=UseOutlines,
            pdfstartview=FitH,
            linktocpage=true,
            colorlinks=true,
            linkcolor=blue,
            plainpages=false
           ]{hyperref}
\usepackage{memhfixc}     % fix interactions between hyperref and memoir
\usepackage{xstring}
\usepackage{tcolorbox}
\usepackage{comment}
\usepackage[active,header=false,handles=false,copydocumentclass=false,generate=cxx20.ext]{extract}
\tcbuselibrary{breakable}

\input{layout}
\input{styles}
\input{macros}
\input{tables}

\makeindex[generalindex]
\makeindex[libraryindex]
\makeindex[impldefindex]

%%--------------------------------------------------
%% macros specific to the Networking TS
\newcommand{\completionsig}{\Fundesc{Completion signature}}
\newcommand{\DEDUCED}{\textit{\texttt{DEDUCED}}}
\newcommand{\nativeref}{see~\ref{socket.reqmts.native}}
\newcommand{\CppXref}[1]{\texorpdfstring{C\kern-0.05em\protect\raisebox{.35ex}{\textsmaller[2]{+\kern-0.05em+}}Std}{C++Std} [#1]}
% Alternative formatting of cross-references, resolving stable name to number.
% Needs include{cxx} and CppXIV from cplusplus/draft/source/macros.tex.
% \input{cxx}
% \newcommand{\CppXref}[1]{\CppXIV \S\cxxref{#1}}

% Add two names to the library index as both #1!#2 and #2!#1
\newcommand{\indexlibrarytwo}[2]{%
\indexlibrary{\idxcode{#1}!\idxcode{#2}}%
\indexlibrary{\idxcode{#2}!\idxcode{#1}}%
}
% Add an extensible implementation entry to the main and library name indices
\newcommand{\indexextensible}[1]{%
\indextext{extensible implementation!\idxcode{#1}}%
\indexlibrary{\idxcode{#1}!extensible implementation}%
}

%%--------------------------------------------------
%% fix interaction between hyperref and other
%% commands
\pdfstringdefDisableCommands{\def\smaller#1{#1}}
\pdfstringdefDisableCommands{\def\textbf#1{#1}}
\pdfstringdefDisableCommands{\def\raisebox#1{}}
\pdfstringdefDisableCommands{\def\hspace#1{}}

%%--------------------------------------------------
%% add special hyphenation rules
\hyphenation{tem-plate ex-am-ple in-put-it-er-a-tor name-space name-spaces}

%%--------------------------------------------------
%% turn off all ligatures inside \texttt
\DisableLigatures{encoding = T1, family = tt*}

% \errorcontextlines 10000

\begin{document}
\chapterstyle{cppstd}
\pagestyle{cpppage}

%%--------------------------------------------------
%% configuration
\input{config}

%%--------------------------------------------------
%% front matter
\frontmatter
\include{front}

%%--------------------------------------------------
%% main body of the document
\mainmatter
\setglobalstyles

\part[Merging Atom into Modules TS]{Merging Atom into the Modules TS}\label{atom}

This section presents the changes to be applied to the Modules TS in order
to merge the features of the Atom proposal, as directed by Evolution at the
Jacksonville 2018 committee meeting.

A note aboute stylistic conventions: as this document describes a ``diff of
a diff,'' the usual convention of using text style for \added{added} and
\removed{removed} text does not work well. In its place, we use block-style
diffs showing the text of the Modules TS before and after this wording is
applied:

\begin{before}
Here is some text from the \Cpp standard
\added{with some additions from the Modules TS}.
\end{before}

\begin{after}
Here is some text from the \Cpp standard
\added{with some additions from the Modules TS}
\added{and some more from the Atom proposal}.
\end{after}

Unchanged text from the Modules TS is retained in this section so that a
complete picture of the ``after'' wording may be obtained by simply ignoring
the ``before'' regions.
\newpage

\include{scope}
\include{references}
\include{definitions}
\include{general}
\include{lexical}
\include{basic}
\include{declarations}
\include{classes}
\include{overloading}
\include{templates}
\include{preprocessor}

%%--------------------------------------------------
%% appendices

%\appendix
%\include{compatibility}

\makeatletter
\immediate\closeout\XTR@out
\immediate\openout\XTR@out=cxx20-dummy.tmp
\makeatother

\part[Merging Modules TS into C++20]{Merging parts of the Modules TS into C++20}\label{atom}

This section presents the changes to be applied to the C++20 working draft in
order to merge the features representing the common subset of version 1 of the
Modules TS and the Atom proposal.

\renewenvironment{before}{\nullfont}{}
\renewenvironment{after}{}{}

\newpage
\input{cxx20.ext}

%%--------------------------------------------------
%% back matter
%\backmatter
%\include{back}


%%--------------------------------------------------
%% End of document
\end{document}
